%% Copyright 2006-2012 Xavier Danaux (xdanaux@gmail.com).
%
% This work may be distributed and/or modified under the
% conditions of the LaTeX Project Public License version 1.3c,
% available at http://www.latex-project.org/lppl/.


\documentclass[11pt,a4paper]{moderncv} 
\usepackage[T2A]{fontenc} 
\usepackage[utf8]{inputenc} 
% possible options include font size ('10pt', '11pt' and '12pt'), paper size ('a4paper', 'letterpaper', 'a5paper', 'legalpaper', 'executivepaper' and 'landscape') and font family ('sans' and 'roman')
\usepackage[unicode]{hyperref}
\definecolor{linkcolour}{rgb}{0,0.2,0.6}
\hypersetup{colorlinks,breaklinks,urlcolor=linkcolour, linkcolor=linkcolour}

% moderncv themes
\moderncvstyle{classic}                        % style options are 'casual' (default), 'classic', 'oldstyle' and 'banking'
\moderncvcolor{blue}                          % color options 'blue' (default), 'orange', 'green', 'red', 'purple', 'grey' and 'black'
%\renewcommand{\familydefault}{\rmdefault}    % to set the default font; use '\sfdefault' for the default sans serif font, '\rmdefault' for the default roman one, or any tex font name
\nopagenumbers{}                             % uncomment to suppress automatic page numbering for CVs longer than one page

% adjust the page margins
\usepackage[scale=0.95]{geometry}
%\setlength{\hintscolumnwidth}{3cm}           % if you want to change the width of the column with the dates
%\setlength{\makecvtitlenamewidth}{10cm}      % for the 'classic' style, if you want to force the width allocated to your name and avoid line breaks. be careful though, the length is normally calculated to avoid any overlap with your personal info; use this at your own typographical risks...


% personal data
\firstname{Вячеслав}
\familyname{Тамарин}
% \title{-разработчик / Python-программист}               % optional, remove the line if not wanted
\address{Санкт-Петербург, Россия}    % optional, remove the line if not wanted
\mobile{+7~(921)~571~4826}                     % optional, remove the line if not wanted
%\phone{+2~(345)~678~901}                      % optional, remove the line if not wanted
%\fax{+3~(456)~789~012}                        % optional, remove the line if not wanted
\email{vyacheslav.tamarin@yandex.ru}                          % optional, remove the line if not wanted
\homepage{github.com/tamarinvs19}                    % optional, remove the line if not wanted
%\extrainfo{additional information}            % optional, remove the line if not wanted
%\photo[110pt][0.01pt]{picture.jpg}                  % '64pt' is the height the picture must be resized to, 0.4pt is the thickness of the frame around it (put it to 0pt for no frame) and 'picture' is the name of the picture file; optional, remove the line if not wanted
%\quote{Some quote (optional)}                 % optional, remove the line if not wanted

% to show numerical labels in the bibliography (default is to show no labels); only useful if you make citations in your resume
%\makeatletter
%\renewcommand*{\bibliographyitemlabel}{\@biblabel{\arabic{enumiv}}}
%\makeatother

% bibliography with mutiple entries
%\usepackage{multibib}
%\newcites{book,misc}{{Books},{Others}}
%----------------------------------------------------------------------------------
%            content
%----------------------------------------------------------------------------------

\begin{document}

\makecvtitle

\section{О себе}
\cvline
  {Навыки}{Ответственность, внимательность, тайм-менеджмент}
\cvline
  {Интересы}{Python-разработка, веб-разработка, telegram-боты, алгоритмы}
\cvline
  {Языки}{Python, JavaScript, SQL, C++, Kotlin}
\cvline
  {Фреймфорки, библиотеки}{Django, FastAPI, PythonTelegramBot, NumPy, Pandas, jQuery, Bootstrap, PostgreSQL, PyTest, Selenium}

\section{Опыт}
\cventry{Июль 2021--настоящее время}{Веб-интерфейс для выбора элективов}{\newline{}учебная практика}{СПбГУ МКН}{\newline{}\url{ https://github.com/tamarinvs19/choosing_electives}}{}
\cvlistitem {
Сайт для выбора элективов студентами факультета Математики и Компьютерных наук на 3--4 курсы.
}
\cvlistitem{Разработал интерфейс для выбора курсов}
\cvlistitem{Реализовал расширяемую систему управления элективами и хранения данных}
\cvlistitem{Python, Django, JavaScript, jQuery, SortableJS, Bootstrap, PostgreSQL}

\cventry{Май 2021--\\Июнь 2021}{ScreenReaderBot}{pet-проект}{}{\newline{}\url{https://github.com/tamarinvs19/screen-reader-bot}}{}
\cvlistitem {
Программа, которая распознает текст на скрине, ищет вхождение этого текста в базе и возвращает совпадения. Также поддерживаются дополнительные функции настройки области поиска текста на изображении и поиск непосредственно по тексту.
}
\cvlistitem{Создал телеграм-бота, реализовал необходимые команды}
\cvlistitem{Написал скрипты для работы с изображением и распознаванием текста}
\cvlistitem{Python, python-telegram-bot}

\cventry{Янв 2021--\\Май 2021}{Foreign Words}{pet-проект}{}{\newline{}\url{https://gitlab.com/tamarinvs19/foreign_words}}{}
\cvlistitem {
Сайт для повторения иностранных слов, определений или формулировок.
}
\cvlistitem{Разработал интерфейс и бэкенд}
\cvlistitem{Покрыл тестами серверную часть}
\cvlistitem{Развернул на Яндекс.Облаке}
\cvlistitem{Python, Django, JS, Bootstrap, PostgreSQL}


\section{Образование}
\cventry{2019--2023}{Санкт-Петербургский Государственный Университет}{факультет Математики и Компьютерных наук}{Современное программирование}{Бакалавр}{}{}


\section{Достижения}
\cventry{2019}{Призер Санкт-Петербургский олимпиады по математике}{}{}{}{}{}
\cventry{2019}{Призер регионального тура Всероссийской олимпиады по математике}{}{}{}{}{}
\cventry{2019}{Призер регионального тура Всероссийской олимпиады по информатике}{}{}{}{}{}


\section{Публикации}
\cvlistitem{Linux. Настройка клавиатуры. \url{https://habr.com/ru/post/486872/}}
\cvlistitem{Слияние списков на python. Сравнение скорости. \url{https://habr.com/ru/post/510970/}}
\cvlistitem{Обзор статьи о SAUNet. \url{https://habr.com/ru/post/666478/}}



\end{document}


%% end of file `template.tex'.


